\documentclass[a4paper,10pt]{article}
\usepackage[utf8]{inputenc}
\usepackage[english] {babel}
\usepackage[T1]{fontenc}
\usepackage{lmodern}
\usepackage{graphicx}
\usepackage{graphics}
\usepackage{ulem}
\usepackage{amssymb}
\usepackage{url}
\usepackage{circuitikz}
\usepackage{subfig}
\usepackage[a4paper]{geometry}
\geometry{hscale=0.7,vscale=0.7,centering}
\usepackage{vmargin}
\usepackage{amsmath}
\usepackage{amssymb}
\usepackage{amsthm}
\usepackage{moreverb}
\usepackage{listings}
\usepackage{qtree}
\graphicspath{{IMAGES/}}
\newtheorem{theorem}{Théorème}[section]
\newtheorem{defi}{Définition}[section]
\newtheorem{propri}{Propriété}[section]
\newtheorem{ex}{Exemple}[section]
\newtheorem{thesis}{Thèse}[section]
\newtheorem{cor}{Corollaire}[section]

\usepackage{color}

\definecolor{gris}{rgb}{0.95,0.95,0.95}


% Param�rage des listings
\lstset{basicstyle=\footnotesize\ttfamily}
\lstset{keywordstyle=\color[rgb]{0,0,1}\bfseries}
\lstset{identifierstyle=\color{black}}
\lstset{commentstyle=\color[rgb]{0,0.4,0}}
\lstset{stringstyle=\color[rgb]{0.7,0,0}}
\lstset{showstringspaces=false}
\lstset{showtabs=false}
\lstset{tabsize=3}
\lstset{extendedchars=true}
\lstset{breaklines=true}
\lstset{postbreak={}}
\lstset{breakautoindent=true}
\lstset{breakindent=0pt}
\lstset{xleftmargin=0.1cm}
\lstset{xrightmargin=0cm}
\lstset{frame=tb,rulecolor=\color[gray]{.4}}
\lstset{captionpos=b}
\lstset{aboveskip=0.5cm,belowskip=0.5cm}
\lstset{numbers=left}
\lstset{columns=fixed}



\newcommand{\code}[1]{\lstinline{#1}}

\setlength{\parskip}{.7em}

\begin{document}


\begin{flushleft}
\includegraphics[scale=0.3]{logo_ist.jpeg}
\end{flushleft}


\vspace{1cm}

\begin{center}{
\scshape{\LARGE Wolves and squirrel : Part 2 }}

\vspace{0.5cm}
\hbox{\raisebox{0.4em}{\vrule depth 0pt height 0.05cm width 16cm}}
{\setlength{\parskip}{0.2cm}

 \Huge
 \bfseries
 \LARGE  Parallel and Distributed Computing
 \\
CPD \\

\vspace{0.2cm}

}
\vspace{0.5cm}

\hbox{\raisebox{0.4em}{\vrule depth 0pt height 0.05cm width 16cm}}
\vspace{2.5cm}


\end{center}

\vspace{3cm}


\vfill
\begin{flushright}
\textbf{Group 30 :}\\
Cappart Quentin (77827)\\
Mikołaj Jakubowski (77610)\\
Paulo Tome (72419)\\

\end{flushright}
\newpage

\setcounter{page}{1}

\renewcommand\thepage{\arabic{page}}


\newpage

\section*{MPI Architecture}

Our program is based on a Master and Servant model.
In this model, we have two sorts of nodes, the master and the servants.
In few words, the master is the coordinator of the work, and the servants will deal only with a sub-board of the total board.
Let's describe all of this more concretely.

\paragraph{MPI Message}
~\\

We have severals type of messages between the master and the servants :

\begin{enumerate}
 \item \texttt{NEW\_BOARD(from:Position, to:Position)} : Message sent by the master to the servants to give them a part of the board.
 \item \texttt{UPDATE\_CELL(c : cell\_t)} : Message sent by the master or by the servants to notify a modification on particular cells.
 \item \texttt{FINISHED()} : Message sent by the master or by the servants to notify the finishing of a particular task.
 \item \texttt{START\_NEXT\_GENERATION(RED or BLACK)} : Message sent by the master to tell the servants to begin the computation
 of a new generation.
\end{enumerate}

\paragraph{Master}
~\\

There is only one master in the program. He's in charge of the coordination of all the work. He does the following actions :

\begin{lstlisting}
Load the world from the input file
Split the world in different parts (One per servant)
Send position of a sub-board to every servant.

for i in 0 -> 2*Nb of generations:
    Send START_NEXT_GENERATION(color) to every node
    Listen for incoming updates and save them
    Count received FINISHED messages
    if(count == Nb of slaves):
		break;
    Send stored updates to all servants
    Send FINISHED to all servants
    
Send FINISHED to all servants // all generations are finished
Listen for UPDATE_CELL and save them to master board
Count FINISHED messages
if(count == Nb of slaves):
    Print output
    Exit
\end{lstlisting}


\paragraph{Servants}
~\\

All the other computers are servants of the master. Each of them receives a sub-board. The distribution of the work follow this idea.
Each servants do the following actions :

\begin{lstlisting}
 Load the world from a file
 Listen for NEW_BOARD message to select a piece
 while true:
    Listen for message
    if(message = FINISHED):
		break;
    elif(message = START_NEXT_GENERATION(genInfo):
    Compute its part of the board
    Send UPDATE_CELL to the master the message with the modified cells
    Send FINISHED to the master
    Listen for UPDATE_CELL messages from master
    Update the cells and resolve conflicts
    Listen for FINISHED message from master
 Sends UPDATE_CELL to the master with all the cells.
 Exit
\end{lstlisting}

\section*{Performances analysis}

The following array recaps the execution time of the OMP version with severals numbers of threads.
\begin{table}[!ht]
\centering
\begin{tabular}{|r||r|r|r|r|}
  \hline
    Instance     & 1 thread   & 2 threads   & 4 threads  & 8 threads  \\
  \hline
    ex3.in       &  0.008     & 0.008        & 0.017       &  0.020 \\ 
  \hline
    world\_10.in &  0.011      &  0.006       & 0.021       & 0.035 \\ 
  \hline
   world\_100.in &  0.111      & 0.070        & 0.068      & 0.1278 \\ 
  \hline
  world\_1000.in &  12.885   & 7.712       & 6.534      &  6.522 \\ 
  \hline
\end{tabular}
\caption{Performances (in sec) for the different instances.}
\end{table}

We did these experiments on a computer with 4 cores with the input \texttt{instance 10 9 8 100}. We
let to the "iteration print" in the code\footnote{If we remove it, the execution time is to fast to do an useful analyze.}.
First, we notice that for the smallest instance, \texttt{ex3.in},
more threads we have, the program is slower. This was expected. Indeed, given that the instance are small, the overhead of the threads (creation, repartition and
synchronization for example), surpass the gain of multithreading.
For the instance \texttt{world\_10.in}, we notice it from 4 threads.
For the biggest instance, \texttt{world\_1000.in}, we see the speedup when we deal with several threads. The speedup per core decreases with the number
of threads\footnote{We gain more from 1 to 2 threads than for 2 to 4 threads.}.
Finally, given a computer with 4 cores, we are not interested in creating more than 1 thread per core.
\\

Now, let's compare the execution time for  world\_1000.in with the ideal speedup\footnote{$S = \frac{T_{serial}}{N_{threads}}$}
with different numbers of threads :



With two threads we are relatively close to the ideal speedup. However the gap grows with more threads. We expected this king of results.
Indeed, there is always overhead when we deal with threads and not all of our code is multithreaded and one thread(master) does no computation, just coordinates computation and communication.
\end{document}

